

\section{The Application}

\emph{Description of our product. Features and the concept behind the UI. Possible improvements and add-ons.}

\subsection{Usability}

MyCourses is able to create a semi-automatic scheduling for programs at universities or schools. It can also provide a scheduling that was run fully-automatic but the recommended usage is to have MyCourses create preliminary versions of a scheduling. These will then be improved by the users via the functionality for collaborative scheduling also provided by MyCourses. This process is best used iterative through several cycles of automatic assignment and manual improvement. To get better results from the automated scheduling users can define requirements and assign those to courses or resources. A requirement that is assigned to a course is regarded as a constraint that either must be met in case of a so called hard-constraint or simply improves the rating of a scheduling in case of a soft-constraint.

In Furthermore MyCourses provides functionality to

\begin{itemize}
\item enrol in courses

\item view your own schedule

\item view schedule for a room

\item import and export data

\item create, read, delete and modify data related to schedules
\end{itemize}
