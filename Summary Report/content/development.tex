\section{Development process}
In the following two sections our development process will be described and reviewed. The first section is a description of our initial ideas and the rationale behind it. The second section describes how our development process actually happened and where we altered it.

\subsection{Intended development process}
As mentioned before the team lacked experience in projects but we knew a bit of software engineering theory. We all agreed that a perfect evaluation of requirements and creation of an appropriate plan was very unlikely. The reasoning for this assumption is again our lack of experience. On the other hand we also saw need for as much structure as we could sensibly apply. This assumption is based on us not working full-time and conducting most of our work isolated. Thus we decided to use an iterative model, expecting many revisions of earlier assessments. To be more precise we decided to follow the spiral model. We expected the structure of defined iterations and tasks for these iterations to compensate our lack of experience in planning. Of course we did not plan to rely blindly on the model to define our process. We rather chose the spiral model as the basis for our own process and assumed that adaption would become necessary. We did not however have a clear concept of what might require adaption and how said adaption might be realized.

\vskip 2ex

When deciding to follow the spiral model we also planned for risk management and quality assurance phases in every iteration. However we missed the fact that the most important aspect of the spiral model is risk management. How exactly this went unnoticed is not clear, but in the result we planned less risk management than appropriate for a spiral model.

\vskip 2ex

Key components of our process were
\begin{itemize}
	\item regular meetings for coordination
	\item communication through e-mail
	\item evaluation of iterations
	\item ticket system for definition and monitoring of tasks
	\item wiki for shared documents
\end{itemize}

\subsection{Actual development process}
As we expected some of our initial ideas were unsuitable or even obstructive. The following paragraphs will discuss where we adhered to our initial plan and where we did not, as well as evaluate our decisions or lack thereof.

\vskip 2ex

As intended we held regular meetings with the whole team to discuss what has been achieved, what could be improved and how the time until the next meeting will be spend. These meetings addressed mostly organizational issues but design and implementation were also discussed. From the end of May until the beginning of October meetings were held on a two-weeks basis. Within our third iteration in October we picked up the frequency and met weekly. This change is due to  integration of parts of our systems with each other, increasing necessity for coordination. We kept meeting weekly thereafter as we found the interval to be fitting our work-cycles well. Due to our workload and our need for orientation at the start we doubt that a weekly cycle would have been appropriate right away. However we could have benefited from starting to meet weekly in August. At that time we often met several days in a row but used it for coding and did neither define nor evaluate tasks clearly. All the time we invested in August and September might have been used more efficient, had we taken the time to review each week.

\vskip 2ex

E-mail was our main medium of communication throughout almost the whole process. As we all study at the same university  this choice might be surprising. Despite living in relative proximity to each other and meeting often at the university, the times we could meet to work on the project were few. Thus an asynchronous communication was a natural choice. A mailing-list was used for organizational issues as well as for artifact related topics. All e-mails were sent to the list even if they were aimed at a single or few receivers. The assumption was that reading e-mails on all the topics would help distributing knowledge to the whole team. On the other hand the high amount of e-mails posed a significant amount of work if read thoroughly. One might also argue that reading isolated informations about topics does not work very well for passive knowledge transfer, so the positive effect might be even smaller. Another problem of our communication was the tendency to mix several unrelated messages into a single e-mail. As a result everyone was forced to read every mail at least fleetingly. Even worse information could sometimes not be found a few weeks later. A clear separation was attempted several times but we failed to adhere to it.

\vskip 2ex 

We did evaluate our iterations at their end and attempted to find weaknesses as well as ways to improve upon them. Positive aspects were discussed as well to gain a more balanced perspective of how we were doing. However success of our evaluations has been moderate as problems, when identified, were tackled ineffectively. On the other hand we did improve upon our process and the execution of it. We just did it whenever someone, often a group of people, noticed something amiss and informed the others. In all cases a discussion ensued and ended in decisions that were far better respected than the ones obtained during official evaluations. We assume this is due to the swift dealing with real problems, leaving only minor or even purely perception based problems for evaluation. We kept evaluating nonetheless as it was a good opportunity for giving feedback.

\vskip 2ex

We decided to use a trac system providing us with tickets and a wiki. Both were used to coordinate the assignment of tasks and gather information about progress and possible problems. The usage of the ticket system was minimal in the beginning and only picked up later, leading to a conducive usage from October. This is also one of the examples where a problem was informally addressed by a member followed by a significant improvement. Prior to this point the problem had been minor, as almost all the work had been done with most of the team present.

During the process we noticed that the spiral model was not flexible enough for our needs and most likely requires more experience than we had. For example we were unable to provide good estimations for amount and time of work and our iterations were actually far too long. Thus we consecutively changed our process, leading to
\begin{itemize}
	\item shorter work-cycles
	\item mostly informal evaluation and feedback
	\item very little risk-management 
\end{itemize}

Regarding risk-management we planned to have a special phase in each iteration dedicated to evaluation and risk-management. During the project we experienced that we did not fully understand how risk-management can be achieved. Our main problem with this point was the lack of good criteria to decide what it means to fail a certain aspect and what an appropriate counter-measure is. For example it could be clear to us that a certain module could not fail, for else the project would fail. Now it would have been necessary to develop a precise plan of action for each point at which said module might fail. What happened however was that people had vague ideas about a certain course of action but not more. This lead to an all but complete abandoning of risk-management.

\subsection{Timeline}
A short overview of the course our project took will be given in the next paragraphs. The goals for a phase were defined after evaluation of the previous phase was completed. Before our project really started we got to know each other and decided which professor should be our contact person at the university.

\begin{description}
\item[First iteration] started on June 20 and comprised of an initial requirement elicitation, design of our software, prototyping of the scheduler and choosing technologies. During most of this time the academic term was still running and only a fraction of our time was devoted to the project. This iteration ended on September 9 and was evaluated the day before. Much of the time spent in this iteration was still focused on gaining orientation. Still it was necessary for us to take this time and find a suitable approach to the work ahead.

\item[Second iteration] started on September 10 and was devoted to producing the scheduling-algorithm, the web-interface and our ORM as well as weave and bakery. Work cycles were short and meetings were held every few days, including a weekend  devoted to coding. From this point on our iterations were reduced to about one month each and the focus was on implementing and testing. Requirements and design were refactored whenever we felt the need to due to new insights or problems. This iteration ended on October 4 and was evaluated the same day.

\item[Third iteration] accordingly started on October 5 but most work was halted until October 16 due to exams and all-day courses at university. Beginning with October 26 weekly two-hour meetings were held at our university in which everyone reported achievements and problems of the last week, as well as set goals for the next week. These weekly meetings have proven to be a good way of coordinating work as well as motivate each other, so they were conducted for the remainder of the project. The main goal and achievement of this iteration was the integration of the scheduler with the web-interface and the connection to the database. Due to this employment of parts of our software many issues were found and fixed. Accordingly we shifted our focus further towards unit-testing and started using a tool to measure code coverage, that is how much of our code is actually being executed  as part of a unit-test. At the same time more  functionality was added to the web-interface. This iteration ended on November 1 and was evaluated the day before.

\item[Fourth iteration] started on November 2 and was primarily focused on the improvement of our web-interface, adding functionality and a unified design. Some improvements to the scheduler-algorithm were made, most notably the introduction of a greedy-algorithm for the set up of the initial population. Work on the Summary Report was also started in this iteration and the first version which was the basis for all further versions was written. The iteration ended on December 5.

\item[Fifth iteration] started on December 6 and was mainly devoted to minor improvements to the scheduler and writing the beta-version of our summary report. One improvement to our database-access-layer which made query-caching available drastically improved scheduling performance by a factor of 8.
\end{description}