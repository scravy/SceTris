\section{Implementation}

\subsection{Scheduler}

The scheduler was implemented as backend of our application and was therefore a core component. After researching possible implementation approaches we decided to implement the basic structure of the scheduler in pair programming. We expected this decision will result in a  well-conceived design and basic implementation. The second reason was two have at least two team members being knowledgeable of this component.

\vskip 2ex

A lot of effort was spent into the implementation of the data-structure modeling a possible schedule solution in order to apply the genetic algorithm in an efficient way. Genetic algorithms operation are based on a data-structure which enables them to be applied. \emph{Crossover} and \emph{Mutation} need properties which can be manipulated.

\vskip 2ex

For these reasons a \emph{Map} was used.

\[Course \mapsto (Room,Time Slot)\]

Applying \emph{Crossover} and \emph{Mutation} means to take direct affect on this map. However this \emph{Map} serves only as an access for the genetic algorithm. Behind it is the data-structure modeling the actual schedule. A \emph{List} of rooms, each having a separate \emph{List} of time slots. Further each time slot has a \emph{List} of Courses as random factors of our algorithm rise possibility having two course at the same time in the same room. This data-structure is illustraed in the following figure:

