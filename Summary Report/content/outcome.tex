
\section{Outcomes and lessons learned}
\subsection{Outcomes}
%Cite from SCORE:  A SCORE report should indicate what was actually accomplished.
We managed to accomplish the most use cases and fulfill the most requirements. Our final web application is ready to be installed with an easy to use installer. It is highly configurable and has a functional web interface. We used FLOSS software consequently and produced a lot of project- and other documentation of any kind along the way.
%can not be done yet -> The web-application we have developed has some limitations due to different reasons \textsl{reasons to come}. It does not do this and that and another feature we should have implemented is not functional at the moment. The features should be easy to add, because they were considered during system design and no interface changes have to be made. <-

Beside our final application we also produced some side products which are worth to be mentioned. These include but are not limited to:
\begin{description}
	\item[junction]: JavafUNCTIONs you missed.
	\item[bakery]: auto generated code reduces errors
	\item[weave]: annotation based web development in Java
	\item[lego]: xsl building blocks making web design easy
\end{description}
%\emph{to be explained asap (as short as possible) by someone who knows how to do that}

\subsection{Lessons learned}
%* Unit Tests are a mixed blessing: On the one hand they help detect errors and keep the code base stable, on the other side it takes time to write them, which lacks in implementing features.
%* Greatly enhanced my knowledge about XSL and web technologies such as HTML and JavaScript
%* Deepened my understanding of the Java programming language and object oriented programming in general
%* Learned AspectJ / aspect oriented programming
%* Recognized how important automatic build tools are (without, developing is a pain)
%* Learned how to write Apache Ant build files (regards automatic build tools)

%\emph{What each of us had learned technology wise and what we have learned about projects.}
For all of us this was the first project of this size and duration. We took the chance to put into practice what we had learned in our software engineering course. Especially things that are not applicable for smaller projects.

We found that project oriented approaches are hard to adhere to in an educational setting. Our team could not meet as often as we wanted and normal university business claimed much more of our project time than we had planned. Nevertheless we managed to establish a systematic meeting schedule. Afterwards it could be noticed that we should have made our iterations shorter in order to better keep track of tasks being postponed.

At the beginning the proposed architecture of our software changed several times. Even with the given set of requirements described. This was due to the hard to understand requirements. There where different interpretations of the described requirements. It took us excessively long to clarify the requirements and gain a shared understanding.

We had to undergo the experience, that a complex development environment has to be easy to set up. Thus the importance of automatic build tools is crucial or developing will be a pain for everyone involved.
Another thing to mention is unit testing. It helps to detect errors and avoid them in code, so making your code base clean and correct. But it also takes a huge amount of time which could be used for implementing features. You have to balance the feeling that you do nothing useful against the advantages of unit testing which is not quite easy depending on the person developing.

Our team had not worked together previous to this project. We experienced that we are very different people with different approaches to problems. Near to the end of this project we are good friends. Everyone learned to get along with the originalities of the others. We have to highlight, that respect and honesty among the members of a team is very important and will hopefully lead to a productive development environment. The project gave us valuable experience on teamwork and how underestimated it is for the success of projects.

During  the  project  we learned  some  new  technologies and kept learning new aspects about well-known technologies.
List of things we learned stated by the team members:
\begin{itemize}
	\item XML, XSL
	\item Apache Ant
	\item aspect oriented programming and therefore AspectJ
	\item enhanced knowledge about web technologies
	\item deepened the understanding of the Java programming language and object oriented programming in general
\end{itemize}
