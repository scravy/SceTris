\end{multicols}
\begin{multicols}{2}[\section{Outcomes \& lessons learned}]
\label{sec:outcomes}

For all of us this was the first project of this dimension and duration. We took the chance to put into practice what we had learned in our software engineering course – in particular going through the whole software development cycle. We found that project oriented approaches are hard to adhere to in an educational setting. Our team could not meet as often as we wanted and our studies interferred much more with our project than we had imagined.

Thus, working on a project like this one was a very challenging task. Sometimes it was frustrating to keep working on the project, due to slow progress or heavy workload. But, it was also a very exciting experience. We could not only learn about many technologies we previously did not know, but we also learned a lot about ourselves and about team work. We experienced that we are very different people with different approaches to problems. Everyone learned to get along with the idiosyncracies of the others.

In particular we learned how to develop software collaboratively and maintain a large code base. We learned a lot about configuration management, build process automatization, web development, and unit testing, which are all topics which we heard about at our university but never got to excercise in practice. We recognized how it is not always easy to write unit tests instead of implementing features as there is the feeling of stalling progress.

Since all of this was a new domain for us, we did not manage to stick to our plan, often due to goals that were set too high to begin with. We realized how important it is to meet regularly and how hard it is to coordinate a group of just five students. The meetings had the the strongest effect on the team. While discussing problems and comparing work results we have learned to give constructive feedback to each other. It is now clear to us, that honesty and respect have a great impact on the success of a team.

However we managed to fulfill the major requirements. We are content with what we achieved, since we did have a lot of fun and the final application is a stable and well working piece of software.
